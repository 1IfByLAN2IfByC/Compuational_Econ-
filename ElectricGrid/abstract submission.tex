\documentclass{article}
\usepackage{fixltx2e}
\usepackage{float}
\usepackage{amsmath}
\newcommand{\degree}{\ensuremath{^\circ}}
\usepackage{graphicx}
\usepackage[margin=1.15in]{geometry}
\usepackage{setspace}
\usepackage{mathpazo}
\usepackage{algorithmic}


\title{Efficient Pricing of Carbon in the EU and its Effect on Consumers}

\author{Michael Lee \\*
The University of Texas at Austin}



\begin{document}
\maketitle{}
\onehalfspacing

A European single market for electricity is modeled to find the optimal portfolio of energy generation technologies in the presence of a carbon tax. The goal is to find the Pareto optimal carbon tax rate such that both carbon emissions and production costs are minimized. Different sources of electricity-- namely coal, natural gas, nuclear, wind, offshore wind, and solar-- are given levelized costs and carbon dioxide emissions ($CO_{2}$) on a per megawatt-hour (MWh) basis. 20,000 energy portfolios, each with different allocations of the respective generation techniques, are generated via a Monte Carlo process and subsequently evaluated by their per MWh cost and emissions. The cost of each generation technology is related to the upfront capital expense, the variable operations and resource costs (O\&M), the amount of $CO_{2}$ it produces and the EU-wide carbon tax rate. This tax-rate is increased until the most cost-efficient portfolio is also the least $CO_{2}$ producing-- thus finding the optimal carbon tax-rate for aligning environmental and economic interests. {\bf Data extracted from this model suggests that this efficient price is around \$80 USD per ton of $CO_{2}$}\*

The effective production price per MWh from the simulation is then compared to the average industrial power price for each of the EU-member states in order to evaluate the effect of an EU-wide carbon tax on end-users. {\bf The optimal portfolio recommended by the simulation, in conjunction with transport via a Pan-European SuperGrid, will be able to supply power at a similar ($\pm 5\%$) price to the current EU 27 average while dramatically reducing greenhouse gas emissions.}  \
*

% Additionally, the current vulnerability of European power to shocks in the supply of natural gas is evaluated and compared to the optimal portfolio suggested by the model. The results show that {\bf current investment in renewable technologies (whether induced by a carbon tax or not) can dramatically mitigate the adverse effects on consumer prices} caused by a Russian-led price increase. \\*

Further research will investigate the optimal location of each power source given transmission losses and spot pricing and availability for requisite resources (e.g. coal, natural gas, average wind speed etc.), as well as the distortionary effects of subsidies in specific nations.

\end{document}